\documentclass[9pt]{amsart}
\usepackage[utf8]{inputenc}
\usepackage{amsmath, amssymb, amsthm}
\usepackage{mathrsfs, graphicx, tikz}
\usepackage[left=3cm, right=3cm, bottom=3.4cm]{geometry}
\usepackage{hyperref}
\usepackage{fancyhdr}

\renewcommand{\qedsymbol}{\emph{(end of proof)}}

\theoremstyle{plain}
\newtheorem{theorem}{Theorem}
\renewcommand{\thetheorem}{\Roman{theorem}}

\newtheorem{definition}{Definition}
\renewcommand{\thedefinition}{\Roman{definition}}

\newtheorem{lemma}{Lemma}
\renewcommand{\thelemma}{\Roman{lemma}}

\newtheorem{conjecture}{Conjecture}
\renewcommand{\theconjecture}{\Roman{conjecture}}


\title{\textbf{On an Erdős Problem}}
\author{Jayant Sharma}
\date{\today}

\pagestyle{fancy}
\fancyhf{}
\fancyhead[L]{\emph{On an Erdős Problem, Jayant Sharma}}
\fancyhead[R]{\thepage}
\renewcommand{\headrulewidth}{0.4pt}
\renewcommand{\footrulewidth}{0pt}

\begin{document}
\maketitle
\begin{abstract}
The Erdős-Straus conjecture was proposed by Paul Erdős and Ernő Straus in 1948. The conjecture asks wether, for every $n\geq 2, \in \mathbb{N}$, we can express,
\[
\frac{4}{n} = \frac{1}{x} + \frac{1}{y} + \frac{1}{z}
\]
Such that, $x, y$ and $z$ are natural numbers.
In this paper, we aim to investigate the conjecture, but generalize it to any $k$ number of fractions. That is, \emph{for any natural number $k$, can we express $\frac{2(k-1)}{n}$ as the sum of reciprocals of $k$ natural numbers.}
For $k=1$ the solution is when $x_1 = x_2 = \frac{n}{2}$, and for $k=2$, the conjecture reduces to the famous \emph{Erdős-Straus Conjecture}. 

In this paper, by the means of *Prime Factor Anlyasis* we show that the conjecture is true, for all $k \in \mathbb{N}$.
\end{abstract}
\tableofcontents

\nocite{*}

\section{Preliminaries}

We begin by formally stating the generalized form of the Erdős–Straus-type conjecture we propose:

\begin{conjecture}[Generalized Erdős–Straus Form]
For any integer $k \in \mathbb{N}$, the rational expression $\frac{2k}{n}$ can be represented as a sum of $(k+1)$ unit fractions. That is, for every pair $(k, n) \in \mathbb{N}^2$ with $n \geq 2$, there exists a set of natural numbers $\{ x_0, x_1, \dots, x_k \} \subset \mathbb{N}$ such that:
\[
\frac{2k}{n} = \frac{1}{x_0} + \frac{1}{x_1} + \cdots + \frac{1}{x_k} = \sum_{i=0}^{k} \frac{1}{x_i}.
\]
\end{conjecture}

We now justify the necessity of the condition $n \geq 2$ by establishing the following lemma:

\begin{lemma}
For every $k \in \mathbb{N}$, the equation in Conjecture~1 admits a solution when $n = 2$.
\end{lemma}

\begin{proof}
Substituting $n = 2$ into the conjectured identity, we aim to express:
\[
\frac{2k}{2} = k
\]
as a sum of $k+1$ unit fractions of natural numbers.

We construct the following multiset $S$ of size $k+1$:
\[
S = \underbrace{\{ 1, 1, \dots, 1 \}}_{k-1\text{ times}} \cup \{ 2, 2 \}.
\]

Clearly, the sum of the reciprocals of the elements in $S$ is:
\[
\left( \sum_{i=1}^{k-1} \frac{1}{1} \right) + \frac{1}{2} + \frac{1}{2} = (k-1) + \frac{1}{2} + \frac{1}{2} = k.
\]

Hence, the equality
\[
k = \sum_{i=0}^{k} \frac{1}{x_i}
\]
holds true for $x_i \in S$, satisfying the conjecture for $n = 2$. This completes the proof.
\end{proof}

One may naturally ask that, \emph{if by Lemma 1, the conjecture holds for $n=2$, then does it also holds for any $n \geq 2, \in \mathbb{N}$?}
This is specifically what builds the \emph{Conjecture 1}.


\bibliographystyle{plain}
\bibliography{document}

\end{document}
